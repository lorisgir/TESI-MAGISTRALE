\chapter*{Abstract}
L’Anomaly Detection é un topic sempre più importante e il suo utilizzo spazia dal campo medico per arrivare a quello finanziario passando anche per approcci più standard come l'analisi di sensori installati su strumenti o macchinari.

Il task che risolve é quello di identificare eventi o osservazioni rari oppure che deviano in maniera significativa dalla maggioranza dei dati e che non corrispondono ad una definizione di comportamento normale. Ricercare queste anomalie può essere utile quando si devono applicare metodi statistici ed una pulizia dei dati é necessaria, ma non solo. In molte applicazioni le anomalie sono di alto interesse in quanto possono contenere informazioni di rilievo e quindi necessitano di attenzione. 

I metodi di Anomaly Detection si dividono tra Supervisionati, Semi-Supervisionati o Non Supervisionati ed un ampio numero di essi sono stati proposti nella letteratura ma non esiste un metodo che sia il più accurato per ogni dataset. Inoltre, la disponibilità di etichette di anomalia per un certo dataset é solitamente bassa o completamente assente nella pratica. 


L’obiettivo di questa tesi é quello applicare metodi di Non Supervisionati di Anomaly Detection all'interno del progetto Beat 4.0 portato avanti da SKF e ALTEN ITALIA. 

A seguito di un'introduzione sul contesto in cui si opera, dei problemi e delle principali tecniche proposte nella letteratura, verrà mostrato un algoritmo di Model Selection che va a rispondere alla seguente domanda: dato un dataset senza etichette ed un set di Anomaly Detectors, come poter selezionare il modello più accurato? A questo scopo vengono definite tre classi di metriche non supervisionate chiamate \textit{Model Centrality}, \textit{Clustering Coefficient} e \textit{Performance on Injected Synthetic Anomalies} e viene mostrato come queste siano correlate rispetto alla metrica supervisionata F1-Score. Saranno proposti anche diversi metodi di Rank Aggregation: Borda, Robust Borda, AVG Score e Kemedy-Young; per combinare insieme le 3 metriche non supervisionate insieme ad un'analisi approfondita sulle performance di ognuno rispetto a dataset di benchmark provenienti da ODDS e SMD.

