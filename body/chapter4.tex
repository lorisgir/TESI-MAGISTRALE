\chapter{Model Selection}

\section{Introduzione}
Model Selection è il processo di selezione di un modello finale, che sia di machine learning o deep learning, tra una serie di modelli candidati per uno specifico dataset.

E' un processo che può essere applicato sia a diversi tipi di modelli, ad esempio regressione logistica, SVM, KNN ecc.; sia a modelli dello stesso tipo ma con configurazione di iper-parametri differenti, ad esempio kernel diversi in un SVM.

Model Selection risulta molto utile ed efficace quando si e' interessati nel sviluppare un modello di classificazione o regressione per un dataset, ma non si sa a priori quale modello funzioni meglio. Di conseguenza la soluzione e' quella di fare training e valutazione per ogni modello preso in considerazione.




\section{Supervised Model Selection}
Le metodologie di Model Selection supervisionato sono quelle classiche che di solito vengono applicate durante lo sviluppo di algoritmi di Machine o Deep Learning. Avendo a disposizione le labels, le scelte ricadono dunque su come partizionare il dataset in train/split e validation set, su come pesare la complessita di un modello e su quali metriche di valutazione adottare.

\subsection{Resampling Methods}
I metodi di resampling, come suggerisce il nome, sono semplici tecniche di partizionamento dei dati e servono per valutare se il modello generalizza bene anche sull'insieme di test. Diversi tipi di split possono essere applicati:
\begin{itemize}
	\item \textbf{Random Split}
	\item \textbf{Time Based Split} 
	\item \textbf{K-Fold Cross Validation}
	\item \textbf{Stratified K-Fold}
\end{itemize}
Un approfondimento di queste tecniche e' stato proposto nel capitolo TODO

\subsection{Ottimizzazione degli iper-parametri}
L'ottimizzazione degli iperparametri è il processo di ricerca del miglior insieme di iperparametri per un modello di machine learning. Gli iperparametri sono dei parametri che non vengono appresi dai dati durante l'addestramento, ma vengono impostati prima dell'inizio di questo. Esempi di iperparametri sono il learning rate di una rete neurale, il numero di alberi in una random forest o regularization factor di un modello lineare. Nell'ambito del model selection, l'obiettivo dell'ottimizzazione degli iperparametri è quello di trovare l'insieme di iperparametri che consente di ottenere le migliori prestazioni per il modello in analisi. Ciò può essere fatto utilizzando tecniche come grid search, random search o l'ottimizzazione bayesiana per esplorare lo spazio degli iperparametri e trovare l'insieme ottimale.


\subsection{Probabilistic Methods}
I metodi probabilistici non tengono conto solo delle prestazioni del modello, ma anche della sua complessità. La complessità del modello è misurata dalla capacità di catturare la varianza dei dati. 
Ad esempio, un modello con un bias alto come l'algoritmo di regressione lineare è meno complesso, mentre una rete neurale ha una complessità molto più elevata.
Un altro punto importante da notare è che le prestazioni del modello prese in considerazione nelle misure probabilistiche sono calcolate solo dal set di train. In genere non è necessario un set di test.
Uno svantaggio, invece, risiede tuttavia nel fatto che i metodi probabilistici non considerano l'incertezza dei modelli e hanno la possibilità di selezionare  quelli più semplici rispetto a quelli complessi.
\begin{itemize}
	\item \textbf{Minimum Description Length} o MDL, deriva dalla teoria dell'informazione che si occupa di metriche come l'entropia, che misura il numero medio di bit necessari per rappresentare un evento da una distribuzione di probabilità o da una variabile casuale. 
	      MDL è dunque il numero minimo di bit necessari per rappresentare il modello.
\end{itemize}

\subsection{Metriche di valutazione}
I modelli possono essere valutati utilizzando diverse metriche, tuttavia, la scelta di una metrica di valutazione consona è cruciale e spesso dipende dal problema da risolvere. Una chiara comprensione di un'ampia gamma di metriche può aiutare a trovare una corrispondenza appropriata tra la descrizione del problema ed una metrica.
Per una descrizione delle metriche utilizzabili nel Model Selection, fare riferimento al capitolo TODO

\section{Unsupervised Model Selection}
Tecniche e metriche sopra descritte hanno applicazione solamente quando si ha a disposizione ground truth labels. Spesso pero', e sopratutto nel dominio dell'Anomaly Detection, le labels non sono disponibili e bisogna quindi cambiare completamente approccio. 
Unsupervised Outlier Model Selection, in breve UOMS, ha ricevuto fin'ora poca attenzione da parte dei ricercatori, tant'è' che solamente una manciata di lavori sono stati pubblicati. Queste proposte assumono diversi approcci al problema e possono essere divisi in due categorie: meta learning e metriche surrogate.

\subsubsection{Definizione del problema}
Sia \({x_t,y_t}^T_{t=1}\) un dataset multidimensionale o una serie temporale multivariata con osservazioni (\(x_1,...,x_T\)), \(x_t\in\Re^d\) e label di anomalia (\(y_1,...,y_T\)), \(y_t \in \{0,1\}\), dove \(y_t=1\) indica che l'osservazione \(x_t\) e' anomala. Le labels saranno usate solamente per la fase di valutazione dei metodi proposti e non per la selezione del modello.

Sia \(M=\{A_i\}^N_{i=1}\) un set di N modelli candidati di anomaly detction dove ogni modello \(A_i\) e' una tupla (\(detector, iper-parametri\)) .
I modelli di anomaly detection non necessitano labels per il training e di conseguenza la fase di train/test split puo essere ignorata.  Questi passaggi vengono comunque eseguiti e consideriamo quindi uno split \(\{x_t\}_{t=1}^{t_{test}-1}\),\(\{x_t\}^{T}_{t=t_{test}}\).

Viene assunto che il modello allenato \(A_i\), quando applicato alle osservazioni \(\{x_t\}^{T}_{t=t_{test}}\) , produca degli score di anomalia \(\{s_t^i\}_{t=t_{test}}^T\),\(s^i_t\in\Re_{\geq0}\). Valori piu alti per gli score di anomalia indicano che l'osservazione e' piu probabile essere anomala.
Le performance di un modello possono essere misurate usando una metrica supervisionata \(Q(\{s_t\}^T_{t=1},\{y_t\}^T_{t=1})\) come F1 Score.

Possiamo ora definire il problema come: date le osservazioni \(X_{test}\) ed un set di modelli \(M=\{A_i\}^N_{i=1}\) allenato usando \(X_{train}\), selezionare un modello che massimizzi la misura di qualita' \(Q(A_i(X_{test}),Y_{test})\) senza utilizzo di label per la selezione.

\subsection{Meta Learning}
Questo approccio mira ad identificare il modello migliore per un particolare dataset date le caratteristiche di questo come il numero di classi, attributi, istanze ecc. L'algoritmo si basa su una collezione di dataset storici di outlier detection in cui le labels sono presenti e sulle performance dei modelli su questi per imparare essenzialmente un mapping tra questi due elementi. 
A questo punto l'algoritmo riceve in input il dataset su cui si vuole fare Model Selection (in cui le labels non sono presenti) ed il risultato di output sarà un modello che l'algoritmo ritiene come migliore sulla base di ciò che ha imparato nella fase di analisi sui dataset storici.
Questo metodo pero' richiede quindi la disponibilità di dataset storici con labels e inoltre fallisce se non ne esiste uno che sia sufficientemente rappresentativo del dataset target.
\subsection{Metriche Surrogate}
A differenza del precedente, questo approccio non ha bisogno di dataset storici o di conoscenza pregressa. Ciò su cui si basa e' la definizione di nuove metriche che non hanno bisogno di labels ma che siano correlate con le più classiche metriche supervisionate (Accuracy, Precision, Recall ecc), da qui il termine metriche surrogate.
La definizione di queste metriche non e' triviale, data proprio la loro caratteristica di essere completamente non supervisionate e possono ricadere in categorie completamente differenti tra di loro. Qualche esempio sono: model centrality o performance on injected synthetic anomalies. Un approfondimento su queste metriche e' presente nel capitolo successivo.


\subsection{Rank Aggregation}
Il ranking è alla base di centinaia di algoritmi come Netflix, Amazon e Google. 
Ad esempio, il Search Index di Google, algoritmo per il ranking delle pagine web a seguito di una ricerca da parte di un utente, combina centinaia di misure di ranking e la combinazione di tali misure avviene in genere con un metodo di Rank Aggregation. 
L'obiettivo del Rank Aggregation è riassumere le informazioni dei singoli ranking di input e fornire un'unica classifica finale, che dovrebbe in qualche modo rappresentare un risultato più accurato o veritiero. 

Se si dovesse formulare il task del Rank Aggregation come un problema di ottimizzazione, come prima cosa e' necessario definire una funzione oggetto. In questo costesto, vorremmo trovare una ranking finale che sia piu vicino possibile a tutte i singoli ranking contemporaneamente. In modo formale, si puo definire la funzione come: \[ \Phi(\delta) = \sum_{i=1}^{m} w_id(\delta,L_i), \]	
dove $\delta$ e' un ranking di lunghezza $k=|L_i|$, $w_i$ e' il peso associato alla lista $L_i$, $d$ e' una funzione di distanza e $L_i$ e' la $i^{ma}$ lista ordinata.
L'obiettivo e' di trovare $\delta^*$ che minimizzi la distanza totale tra $\delta^*$ e gli $L_i$
\[ \delta^* = arg min \sum_{i=1}^{m} w_id(\delta,L_i). \]

Selezionare la funzione di distanza piu appropriata e' molto importante e la scelta di solito ricade alla distanza di Kendall.
Intuivamente, la distanza di Kendall viene definita come la sommatoria, per ogni possibile coppia di elementi date due liste in input, della seguente penalita:
se due elementi $t$ e $u$ hanno lo stesso ordinamento in entrambe le liste, allora nessuna penalita viene data. Altrimenti se $t$ precede $u$ in una lista e $u$ precede $t$ nell'altra, allora viene imposta una penalita di 1.
Questa distanza puo assumere valori compresi nell'intervallo $[0,n(n-1)/2]$ e non e' da confodnere con il coefficiente di Kendall. Quest'ultimo, essendo un coefficiente, assume valori nell'intervallo $[-1,1]$ ed e' prevalentemente usato in statistica. Sono due concetti differenti ma correlati da:
\[K_c=1-4K_d/(n(n-1)), K_d = (1-K_c)(n(n-1))/4\]
Quindi quando $K_c=1$ il valore di $K_d$ e' 0, al contrario quando $K_c=-1$ il valore di $K_d$ e' massimo.

Il problema di Rank Aggregation come sopra definito viene anche chiamato "Kemeny-Young problem". Purtroppo e' un problema NP-Hard anche con un numero di ranking basso come 4. Per questo motivo, in questa tesi verranno presentati e usati anche metodi alternativi che approssimino la soluzione in modo più efficiente, come ad esempio Borda. Dettagli su questi metodi saranno esposti nei capitoli successivi.

\subsubsection{Rank Comparison}
Ottenuti i ranking di ogni metrica surrogata e generato il rank finale dopo l'aggregazione, per valutare le performance dell'algoritmo di Model Selection e' necessario comparare i risultati prodotti rispetto ad un rank di riferimento.
In questa tesi saranno usati tre indici differenti applicati o sui ranking di posizione oppure sui ranking contenti gli score prodotti dal Model Selection e saranno:

\begin{itemize}
	\item \textbf{Kendall}: coefficente che usa la distanza di Kendall sopra descritta. Usato sui ranking di posizione
	\item \textbf{Spearman}: usato anche sui ranking di posizione, questo metodo statistico quantifica il grado in cui le variabili sono associate da una funzione monotona, indicando quindi una relazione crescente o decrescete. Viene usato su variabili cardinali e tiene conto dei rank piuttosto che dei dati grezzi.
	\item \textbf{Pearson}: usato sui ranking di score, a differenza di Spearman viene usato solamente su variabili continue e misura una correlazione lineare tra le due variabili.
\end{itemize}
